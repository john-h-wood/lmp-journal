\documentclass{article}
%Last edited December 21, 2022

% ===========================================================
% Package imports
% ===========================================================
\usepackage{amssymb}
\usepackage{mathtools}
\usepackage{fancyhdr}
\usepackage{titling}
\usepackage{lastpage}
\usepackage{siunitx}
\usepackage{physics}
\usepackage{float}
\usepackage{amsthm}
\usepackage{bm}
\usepackage{physics}
\usepackage{tikz}

\usepackage[  
colorlinks=true,
linkcolor=blue
]{hyperref}

% ===========================================================
% Doucment formatting
% ===========================================================
% Page layout and spacing
\usepackage[letterpaper, margin = 1in]{geometry}
\setlength{\parindent}{0pt}
\setlength{\parskip}{8pt plus2pt minus2pt}

% Default author, title, and date
\title{Default Title}
\author{John Wood}
\date{\today}

% Set page numbering to  'Page x of y' format
\pagestyle{fancy}
\fancyhf{}
\renewcommand{\headrulewidth}{0mm}
\fancyfoot[C]{Page \thepage \hspace{1pt} of \pageref{LastPage}}

% Title page style (same footer, but head rule width is zero)
\fancypagestyle{titlepagestyle}{
	\fancyhf{}
	\renewcommand{\headrulewidth}{0mm}
	\fancyfoot[C]{Page \thepage \hspace{1pt} of \pageref{LastPage}}
}

% Three place Fancy header and footer
\newcommand{\header}[3]{ %add optional head-rule width paramter
	\renewcommand{\headrulewidth}{0.5pt}
	\fancyhead[L]{#1}
	\fancyhead[C]{#2}
	\fancyhead[R]{#3}
}

% Bolded title
\newcommand{\boldtitle}{
	\thispagestyle{titlepagestyle}
	\begin{center}
		\textbf{\Large{\thetitle}}
	\end{center}
}

% ===========================================================
% New commands
% ===========================================================
% Sets
\newcommand{\R}{\mathbb{R}}
\newcommand{\N}{\mathbb{N}}
\newcommand{\Z}{\mathbb{Z}}
\newcommand{\Q}{\mathbb{Q}}
\newcommand{\C}{\mathbb{C}}
\renewcommand{\emptyset}{\varnothing}
\renewcommand{\epsilon}{\varepsilon}

% Logic one
\newcommand{\limplies}{\rightarrow}
\newcommand{\limpliedby}{\leftarrow}
\newcommand{\liff}{\leftrightarrow}

% Reference
\newcommand{\reff}[1]{\hyperref[#1]{(\ref*{#1})}}
\newcommand{\use}[1]{\text{by } \reff{#1}}

% Proof
\newcommand{\st}{\text{ s.t. }}
\renewcommand{\qed}{\begin{flushright}$\blacksquare$\end{flushright}}

% Operators
\DeclareMathOperator{\img}{img}
\DeclareMathOperator{\cod}{cod}
\DeclareMathOperator{\dom}{dom}
\DeclareMathOperator{\idx}{idx}

% Other
\newcommand{\ds}{\displaystyle}

% ===========================================================
% Theorems
% ===========================================================
\newtheorem{definition}[equation]{Definition}

\theoremstyle{remark}
\newtheorem*{remark}{Remark}
\newtheorem{theorem}{Theorem}
\title{Positional Notation}
\DeclareMathOperator{\val}{val}


\begin{document}
	\boldtitle
	
	% TODO define symbol
	% TODO add motivation
	
	\section{Numbers and Notation}
	We begin by trying to justify a distiction between numbers and their notation. If you are asked to write a random number, you might write ``10.'' That peice of writing is not, itself, the number you are thinking of. Rather, the symbols you write \textit{refer} to a certain number. By ``symobls,'' I mean certain shapes which are given meaning. Letters, and other glyphs are all symbols. To see this reference-base relationship between symbols and numbers, consider that the same number could be written differently. Using Roman numerals, the symobl ``X'' refers to the same number as ``10.'' Alternatively, notice that different languages use unique speech to speak of the same number. We don't say that an enligh-speaker's ``ten'' and a french-speaker's ``dix'' are different numbers. Instead, we know that English and French have different methods of referring to the same number.
	
	So, there is a difference between a number and how it is written and said. When a number is written or spoken, there is a reference to the number. A system for writting representations of something is sometimes called a ``notation.''
	
	We can think of the reference-based relationship between notations and numbers with functions. For example, consider the set of symbols $\mathbb{S}$ of some notation which represents real numbers. For this notation to be useful, there must be some function $f:\mathbb{S}\rightarrow\R$ which gives the number refered to for each symbol.
	
	Here, we invesitgate positional notation.
	
	\subsection{Notation Requirement}
	In our case, it is impossible to consider numbers without notation systems. As I write equations here, I must use some writing to insert numbers into your mind. When I write ``0,'' you read that as zero and then the number appears in your mind. There is no other way to do this in writing other than to use a system which you and I both understand. I commonly use what is called the decimal system.
	
	Here, I intend to dicsuss symbols such as ``0'' separately from the numbers they bring up in your mind; the numbers they reference. But then, how should I bring up the actual number referenced by ``0?'' To do so, let us define a new notation for numbers. This will be the same system as we have always used, but with bars on top of the symbols. For example, the symbols $\overline{34}$ refer to the same number as $34$. I will carefully use this system for when a symbol should refer to a number. Thus, when writing ``45,'' I am not trying to refer to any number, but am just writing certain symbols.
	
	\section{The Decimal System}
	One approach to discuss positional notation is to outright define it. In this section, I add some motivaion to the definition. We will analyze how we write numbers to get some insight in the sytem we use.
	
	Consider writing down numbers successively. Only worrying about the natural numbers $\N$, we start with zero ($\overline{0}$) and write each number on a sheet of paper. The first few numbers are written with single symbols:
	
	\begin{equation}\label{eq1}
		\pqty{0, 1, 2, 3, 4, 5, 6, 7, 8, 9}
	\end{equation}
	
	
	Once we have reached nine, how do we write the next number? We combine a ``1'' with a ``0.'' Next, we keep the first ``1,'' and iterate the second symbol through the sybols in \reff{eq1}:
	
	\begin{equation}\label{eq2}
		\bm{1}0\to \bm{1}1\to\bm{1}2\to\dots\to\bm{1}9
	\end{equation}
	
	After we exaust all posiibilities for the second symbol, we increment the first:
	
		\begin{equation}\label{eq3}
		\bm{2}0\to \bm{2}1\to\bm{2}2\to\dots\to\bm{2}9
		\end{equation}
	 
	And so on until we reach ``99,'' at which point we add a third symbol and return the others to ``0'':
	
		\begin{equation}\label{eq4}
		\bm{9}9\to\bm{10}0\to\bm{10}1\to\bm{10}2\to\dots
		\end{equation}
	 
	Notice now that as we continue, we keep the first two symbols in place until we exaust all possibilities for the last symbol.
	
	In general, we start with a single symbol that incrments through the possibilities in \reff{eq1}. Then, add a symbol in front and keep it in place as the second symbol increments. Once a symbol has gone through its possibilities, work on the one before it. If there isn't one before, it, add one.
	
	Notice that there are $\overline{10}$ symbols in \reff{eq1}. So, once we've exausted a single symbol, we have gone through $\overline{10}$ elements of the natural numbers. That is, we have written symbols for the numbers from $\overline{0}$ to $\overline{9}$. Adding a second symbol in our writing gets us to the number $\overline{10}$. For each possibility of the new symbol, we cycle through each of the $\overline{10}$ possibilities from \reff{eq1}, so that ``99'' represents the $\pqty{\overline{10}+\overline{9}\cdot\overline{10}}^\text{th}$ element $\N$ (the number $\overline{99}$ since $\N$ starts with $\overline{0}$).
	
	When we get to three symbols, each increment of the first has $\overline{100}$ possibilities for the second and third symbols (``00'' to ``99''). This follows from the previous paragraph, with the addition of a leading ``0'' to the first $\overline{10}$ number representations. Thus, ``x00,'' where is one of our symbols, represnts the number $\overline{x}\cdot\overline{100}$.
	
	%TODO address leading zeros
	
	We can generalize from here. The addition of a new symbol gives the chance to cycle through the possibilites of the other symbols with a leading ``0.'' For example, to get to ``4321'', we've gone through each possibility of the first symbol from 1 to 4, each possibility of the second symbol from 1 to 3, and so on. 
	
	Since we are working with ten symbols, each additional symbol in a representation multiplies the possible representations by ten.
	
	This exploration has shown us that in our common notation system, we use the order of symbols to represent greater and greater numbers. Us having used ten symbols is an arbitrary descision, though. This is addressed by the definitions in the next section.
	
	\section{Positional Notation}
	\subsection{Definitions}
	In this section, we give a more rigorous definition of positional notation. We start by defining numeral systems in general. To do that, we must define glyphs. This is a rather philosohpical task. In this context, we take a glyph to mean something like a written ``2'' or that same shape on a screen. We are focused mainly on writting, but even a spoken word can be a glyph. This allows combinations of words to represent numbers in the same was as written glyphs.
	
	\begin{definition}\label{def5}
		A glyph is an object.
	\end{definition}
	\begin{remark}
		This is an extremely broad defintion, but I find it necessary when considering, in the broadest of terms, what can be a glyph. In most cases, glyphs are written (or displayed, for computer screens and in general) or spoken objects. When displayed, a glyph is a goemtric shape that has certain distinguishing characteristics (e.g. the circular nature of ``0''). When spoke, a glyph is a series of pressure waves in a medium that, when they affect an ear drum, are understood, given some language knowledge of the hearer.
		
		These cover most cases, but any object could be a glyph. There is nothing stopping us from calling, say, a dog a glyph. The dog, in a certain numeral system, might repsresent the number $\overline{22}$. Even a dream could be chosen to represent a certain number.
	\end{remark}
	
	\begin{definition}\label{def 6}
		A numeral system for a set of numbers $X$ is a tuple
		$$\pqty{G, V}$$
		where $G$ is a set of glyphs and $V$ is a function from non empty tuples of elements of $G$ to $X$.
	\end{definition}
	
	With the above, we can now define positional notation. This will be a long defintion, so we first dicuss. The idea behind positional notation is to have numbers represted with ordered series made of a possible $b\in\N^*$ glyphs. Each positional is to be `worth' exponential values of the base. For example, $xyzw$ in base $b$ would represent the number
	\begin{equation}\label{eq7}
		w\cdot b^{\overline{0}} + z\cdot b^{\overline{1}} + y\cdot b^{\overline{2}} + x\cdot b^{\overline{3}}
	\end{equation}
	For this equation to be sensical\footnote{Sensical without defining addition and multiplication between $b$ and symbols. The most important part of the positional system, at least that which is commonly taught, is how the position of glyphs contributes differents exponents of the base to the final value. However, how each glyph references a number is a vital part of the system, making it a numeral system.}, we would need to know the value of the symbols in $xyzw$. The positional system assigns numbers to each of the $b$ glyphs using their position in a tuple. To accomplish this, we will define an index function,  which gives us the position of an element in a tuple. This
	
	Let $T$ be an $n$-ary tuple with $n\in\N^*$ without repeating elements. Then for any $x\in T$, the index function if a function
	
	\begin{equation}\label{eq8}
		\idx: \Bqty{T}\cross x\to \Bqty{\overline{0}, ..., n-\overline{1}}
	\end{equation}
	
	defined by
	
	\begin{align}\label{eq9}
		\idx (T, x) &= \text{the position of $x$ in $T$ with} \\
					&\qquad\text{the first element having position $\overline{0}$}\nonumber \\
					&\qquad\text{and the last element having position $\overline{n-1}$}\nonumber
	\end{align}
	
	Notice that if $T$ had repeating elements, indexing would be less intuitive. For example, the position of $1$ in $(2, 9, 1, 0, 7, 1)$ could either be $\overline{2}$ or $\overline{5}$. Avoiding this allows us to identify a \textit{unique} index. For positional notation this in turn allows for a single number to represented by a string of glyphs.
	
	Now, say our base $b$ positional notation has a tuple $T$ of $b$ glyphs. \ref{eq7} would more properly be written
	
	\begin{equation}\label{eq10}
		\idx(T, w)\cdot b^{\overline{0}} + \idx(T, z)\cdot b^{\overline{1}} + \idx(T, y)\cdot b^{\overline{2}} + \idx(T, x)\cdot b^{\overline{3}}
	\end{equation}
	
	The last step is to cover non-whole numbers. You might see that a system allowing only multiples of whole powers of $b$ would only be able to give us whole numbers. How, for example, would we repsent $\overline{0.5}$? The closest we could get right now is $\overline{0}\cdot b^{\overline{0}} = 0$. The solution taken by the positional system is to allow for negative powers of $b$. This allows us to consider fraction values. Glyphs associated with negative powers are delineated with a decimal point: ``.''  Gylphs after the point are associated with negative powers. For example, the representation $xy.tu$ would represent the number
	
	\begin{equation}\label{eq11}
		\idx(T, x)\cdot b^{\overline{1}} + \idx(T, y)\cdot b^{\overline{0}} + \idx(T, t)\cdot b^{\overline{-1}} + \idx(T, u)\cdot b^{\overline{-2}}
	\end{equation}
	
	Notice that from left to right, the power of $b$ decreases by $\overline{1}$ for each glyph. The glyph for the zeroth power is right before the decimal point. For simplicity, we can say that a representation witout a decimal point represents the same number as if a decimal point were at the end.
	
	\begin{definition}
		Let $b\in \Bqty{2, 3, \dots}$. Let $G$ be a set of $b$ glyphs and let $T$ be a tuple of those same glyphs, with some particular order. Then, ``base $b$ positional notation'' is  a numeral system for the set of numbers $\Bqty{x\in\R : x\geq 0}$ with the set of glyphs $G\cup \Bqty{.}$, with $V$ defined by
		
		\begin{equation}\label{eq12}
			\begin{split}
				V(p_{n}, p_{n-\overline{1}}, \dots, p_{\overline{0}})&=V(p_{n}, p_{n-\overline{1}}, \dots, p_{\overline{0}}, .)\\
				V(p_{n}, p_{n-\overline{1}}, \dots, p_{\overline{0}}, ., p_{\overline{-1}}, \dots, p_{d}) &= \sum_{i=d}^{n}\idx(T, p_i)b^i
			\end{split}
		\end{equation}
	\end{definition}
	
	We make some remarks on this definition:
	
	\begin{remark}
		The definition does not specify what glyphs should be used. We commonly use the arabic numerals for base ten positional notation: $(0, 1, 2, 3, 4, 5, 6, 7, 8, 9)$, but any glyphs work. Base $b$ positional notation is usually considered seperately from the exact glyphs used. This is reflected in us defining a term without reference to $T$: ``base $b$ positional notation.''
	\end{remark}
		
	\begin{remark}
		The restriction that $b\geq 2$ is necessary since $\overline{1}$ to any power is equal to $\overline{1}$. If we had $b=\overline{1}$, the definition of $V$ be such that non-whole numbers could not be represented. This is supposed to be accomplished by negative powers of $b$ yielding fractional values to use in \ref{eq12}. Since $\overline{1}$ does not satisfy this, the equation contains purely whole numbers.
	\end{remark}
	
	Base ten positional notation is commonly called the ``decimal system.'' There are, in fact, many common bases used with positional notation. We list three with their most commonly used gylphs.
	
	\begin{equation}
		\begin{array}{c | c | c}
			\text{Base} & \text{Name} & \text{Common glyphs}\\
			\hline
			\overline{2} & Binary & \pqty{0, 1}\\
			\hline
			\overline{10} & Decimal & \pqty{0, 1, 2, 3, 4, 5, 6, 7, 8, 9}\\
			\hline
			\overline{16} & Hexadecimal & \pqty{0, 1, 2, 3, 4, 5, 6, 7, 8, 9, A, B, C, D, E, F}
			
		\end{array}
	\end{equation}
	
	For more pratical use, we define two notions:
	
	\begin{definition}
		Positional notation the set of positional notations in all bases.
	\end{definition}
	\begin{remark}
		This clarifies use of the term positional notation. We might say ``using/with/because of positional notation'' to be base-agnostic.
	\end{remark}
	
	\begin{definition}
		In the context of referencing a number, a string of glyphs $s$ ending with a suffix string of glyph representing the number $b$ means that $s$ is to be understood as a represention in base $b$ positoinal notation. The notation used to find the number represented by the suffix is to be determined in context.
	\end{definition}
	\begin{example}
		``$123_{\overline{10}}$'' is a representation in base ten positional notation. I use the overline system to show how to intepret the suffix glyphs.
	\end{example}
	\begin{example}
		$101_2\neq 101_{10}$. Without explicit notation information, you are meant to understand that the suffixes are in base-ten positional notation. 
	\end{example}
	
		
	From the previous examples, you may see that assumptions are required when the numeral system in use is not stated. Very commonly, base ten positional notation is used. In my writing, assume it is unless otherwise specified. From here, we drop the overline notation. When glyphs are used, treat them within the decimal system unless otherwise states.
	
	\subsection{Theorems}
	
	
\end{document}