\documentclass{article}
%Last edited December 21, 2022

% ===========================================================
% Package imports
% ===========================================================
\usepackage{amssymb}
\usepackage{mathtools}
\usepackage{fancyhdr}
\usepackage{titling}
\usepackage{lastpage}
\usepackage{siunitx}
\usepackage{physics}
\usepackage{float}
\usepackage{amsthm}
\usepackage{bm}
\usepackage{physics}
\usepackage{tikz}

\usepackage[  
colorlinks=true,
linkcolor=blue
]{hyperref}

% ===========================================================
% Doucment formatting
% ===========================================================
% Page layout and spacing
\usepackage[letterpaper, margin = 1in]{geometry}
\setlength{\parindent}{0pt}
\setlength{\parskip}{8pt plus2pt minus2pt}

% Default author, title, and date
\title{Default Title}
\author{John Wood}
\date{\today}

% Set page numbering to  'Page x of y' format
\pagestyle{fancy}
\fancyhf{}
\renewcommand{\headrulewidth}{0mm}
\fancyfoot[C]{Page \thepage \hspace{1pt} of \pageref{LastPage}}

% Title page style (same footer, but head rule width is zero)
\fancypagestyle{titlepagestyle}{
	\fancyhf{}
	\renewcommand{\headrulewidth}{0mm}
	\fancyfoot[C]{Page \thepage \hspace{1pt} of \pageref{LastPage}}
}

% Three place Fancy header and footer
\newcommand{\header}[3]{ %add optional head-rule width paramter
	\renewcommand{\headrulewidth}{0.5pt}
	\fancyhead[L]{#1}
	\fancyhead[C]{#2}
	\fancyhead[R]{#3}
}

% Bolded title
\newcommand{\boldtitle}{
	\thispagestyle{titlepagestyle}
	\begin{center}
		\textbf{\Large{\thetitle}}
	\end{center}
}

% ===========================================================
% New commands
% ===========================================================
% Sets
\newcommand{\R}{\mathbb{R}}
\newcommand{\N}{\mathbb{N}}
\newcommand{\Z}{\mathbb{Z}}
\newcommand{\Q}{\mathbb{Q}}
\newcommand{\C}{\mathbb{C}}
\renewcommand{\emptyset}{\varnothing}
\renewcommand{\epsilon}{\varepsilon}

% Logic one
\newcommand{\limplies}{\rightarrow}
\newcommand{\limpliedby}{\leftarrow}
\newcommand{\liff}{\leftrightarrow}

% Reference
\newcommand{\reff}[1]{\hyperref[#1]{(\ref*{#1})}}
\newcommand{\use}[1]{\text{by } \reff{#1}}

% Proof
\newcommand{\st}{\text{ s.t. }}
\renewcommand{\qed}{\begin{flushright}$\blacksquare$\end{flushright}}

% Operators
\DeclareMathOperator{\img}{img}
\DeclareMathOperator{\cod}{cod}
\DeclareMathOperator{\dom}{dom}
\DeclareMathOperator{\idx}{idx}

% Other
\newcommand{\ds}{\displaystyle}

% ===========================================================
% Theorems
% ===========================================================
\newtheorem{definition}[equation]{Definition}

\theoremstyle{remark}
\newtheorem*{remark}{Remark}
\newtheorem{theorem}{Theorem}
\title{Positional Notation}
\DeclareMathOperator{\val}{val}


\begin{document}
	\boldtitle
	
	% TODO define symbol
	
	\section{Numbers and Notation}
	We begin by trying to justify a distiction between numbers and their notation. If you are asked to write a random number, you might write ``10.'' That peice of writing is not, itself, the number you are thinking of. Rather, the symbols you write \textit{refer} to a certain number. To see this, consider that the same number could be written differently. Using Roman numerals, the symobl ``X'' refers to the same number as ``10.'' Alternatively, notice that different languages use unique speech to speak of the same number. We don't say that an enligh-speaker's ``ten'' and a french-speaker's ``dix'' are different numbers. Instead, we know that English and French have different methods of referring to the same number.
	
	So, there is a difference between a number and how it is written and said. When a number is written or spoken, there is a reference to the number. A system for writting representations of something is sometimes called a ``notation.''
	
	We can think of the reference-based relationship between notations and numbers with functions. For example, consider the set of symbols $\mathbb{S}$ of some notation which represents real numbers. For this notation to be useful, there must be some function $f:\mathbb{S}\rightarrow\R$ which gives the number refered to for each symbol.
	
	Here, we invesitgate positional notation.
	
	\subsection{Notation Requirement}
	In our case, it is impossible to consider numbers without notation systems. As I write equations here, I must use some writing to insert numbers into your mind. When I write ``0,'' you read that as zero and then the number appears in your mind. There is no other way to do this in writing other than to use a system which you and I both understand. I commonly use what is called the decimal system.
	
	Here, I intend to dicsuss symbols such as ``0'' separately from the numbers they bring up in your mind; the numbers they reference. But then, how should I bring up the actual number referenced by ``0?'' To do so, let us define a new notation for numbers. This will be the same system as we have always used, but with bars on top of the symbols. For example, the symbols $\overline{34}$ refer to the same number as $34$. I will carefully use this system for when a symbol should refer to a number. Thus, when writing ``45,'' I am not trying to refer to any number, but am just writing certain symbols.
	
	\section{The Decimal System}
	
	
	
	 
\end{document}